\documentclass[10pt,pdf,hyperref={unicode}]{beamer}
\mode<presentation>
\usetheme{AnnArbor}
\usecolortheme{beaver}
\usepackage[utf8]{inputenc}
\usepackage[english,russian]{babel}
\usepackage{amssymb}
\usepackage{graphicx}
\usepackage{epstopdf}
\usepackage[english]{isodate}
\usepackage[noend]{algorithmic}
\usepackage{algorithm}
\usepackage[T2A]{fontenc}

\title[Strategy-profness versus...]{Strategy-proofness versus Efficiency in Matching with Indifferences: Redesigning the NYC High School Match}
\author[БЕЗЕР]{Бажанов К.Н., Егурнов А.А., Зрянина М.С., Егурнов Д.А., Рыжакова Е.В.}
\institute[НИУ ВШЭ]{Национальный Исследовательский Университет Высшая Школа Экономики}
\date{14 Марта 2013}

\begin{document}

\begin{frame}
    \titlepage
\end{frame}

\begin{frame}
    \frametitle{Содержание}
    \tableofcontents
\end{frame}

\AtBeginSection[]
{
    \begin{frame}
        \frametitle{Содержание}
        \tableofcontents[currentsection]
    \end{frame}
}

\AtBeginSubsection[]
{
    \begin{frame}
        \frametitle{Содержание}
        \tableofcontents[currentsection, currentsubsection]
    \end{frame}
}

\section{Об авторах}

\subsection{Atila Abdulkadiroglu}

\subsection{Parag Pathak}

\subsection{Alvin E. Roth}

\section{Введение}

\section{Постановка задачи}

\begin{frame}
    \frametitle{Причина возникновения задачи}
    \begin{block}{Причины}
        \begin{itemize}
            \item Огромное количество школ и абитуриентов\\В 2003-2004 более 90000 абитуриентов.
            \item Старый {\bf децентрализованный} механизм
            \item Школы имели склонность занижать {\bf квоты}
        \end{itemize}
    \end{block}
\end{frame}

\begin{frame}
    \frametitle{Особенности старой модели}
    \begin{block}{Ограничения на предпочтения}
        \begin{itemize}
            \item Школьники могут выбрать только 12 школ
            \item Многие школы ранжируют школьников {\bf пассивно}
        \end{itemize}
    \end{block}

    \begin{block}{Всего три тура}
        \begin{itemize}
        \item В первом туре участвуют только спецшкольники
        \item Оставшиеся места разыгрываются во втором (основном) туре
        \item Нераспределенные школьники могут выбрать ещё 12 школ
        \end{itemize}
    \end{block}
\end{frame}

\begin{frame}
    \frametitle{Модель}
    \begin{itemize}
        \item $I$ -- {\bf множество учеников}
        \item $S$ -- {\bf множество школ}
        \item $q_S$ -- количество мест в школе ({\bf квота})
        \item $P_i$ -- отношение строгого порядка на множестве $S \cup \left\{i\right\}$ (задаёт {\bf предпочтения ученика} $i$)
        \item Школа $s$ {\bf допустима} для ученика $i$, если $s P_i i$.
        \item $R_i$ -- отношение слабого порядка на множестве $I \cup \left\{s\right\}$ (задаёт {\bf предпочтения} $s$)
        \item Ученик $i$ {\bf допустим} для школы $s$, если $i R_s s$.
    \end{itemize}
\end{frame}

\begin{frame}
    \frametitle{Модель}
    \begin{itemize}
        \item Паросочетание $\mu$ {\bf индивидуально рационально} если каждому агенту сопоставляется допустимый агент
        \item Пара $(i, s)$ {\bf блокирующая} если $s P_i \mu(i)$ и либо $|\mu(s)| < q_s$ и $i R_s s$, либо  $\exists i^\prime \in I: i R_s i^\prime$
        \item Паросочетание {\bf доминирует} $\nu$ если $\forall i \in I: \mu(i) R_i \nu(i)$ и $\exists i \in I: \mu(i) P_i \nu(i)$
        \item $\mu$ {\bf стабильно} если оно {\bf индивидуально рационально} и не имеет {\bf блокирующих пар}
        \item $\mu$ {\bf оптимально для абитуриента} если оно не доминируется никаким другим стабильным паросочетанием
        \item $\mu$ {\bf эффективно} если оно не доминируется никаким другим паросочетанием
    \end{itemize}
\end{frame}

\begin{frame}
    \frametitle{Модель}
    \begin{definition}
        $\phi: (P_I, R_S) \longmapsto \mu$ -- {\bf прямой механизм} 
    \end{definition}

    \begin{definition}
        Механизм $\phi$ {\bf dominant strategy incentive compatible (DSIG)} для школьника $i$ если $\forall (R_I, R_S), \forall P_i^\prime: \phi_i(P_I; R_S) R_i \phi(P_i^\prime, P_{-i}; R_S)$
    \end{definition}

    \begin{definition}
        Механизм называется {\bf неманипулируемым (strategy proof)} если DSIG выполняется для всех участников.
    \end{definition}
\end{frame}

\end{document}
